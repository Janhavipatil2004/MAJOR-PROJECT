%%%%%%%%%%%%%%%%%%% use for abstract
%%% refer q31.tex
\begin{abstract}
	This abstract provides a concise summary of the main points and findings of your research paper, thesis, or article. It should be clear, informative, and engaging, allowing readers to quickly understand the purpose and significance of your work.
	
	\textbf{Introduction:} Start by introducing the topic and providing context for your study.
	
	\textbf{Objective/Research Question:} State the main objective of your research or the specific research question you aimed to address.
	
	\textbf{Methodology:} Briefly describe the methods or approaches used to conduct your research. This may include experimental techniques, data collection methods, or theoretical frameworks.
	
	\textbf{Results/Findings:} Summarize the main findings or outcomes of your study. Highlight the key results that contribute to addressing your research question.
	
	\textbf{Conclusion/Implications:} Conclude with the significance of your findings and discuss any potential implications or applications of your research.
	
	This abstract should be concise, typically between 150-250 words, and written in clear, direct language. Avoid unnecessary jargon or complex terminology.
	
	Keywords: Include relevant keywords that reflect the main topics and themes of your research. This helps with indexing and searchability.
\end{abstract}

